%%%%%%%%%%%%%%%%%%%%%%%%%%%%%%%%%%%%%%%%%%%%%%%%%%%%%%%%%%%%%%
%\documentclass[11pt,reqno]{amsproc}
\documentclass{amsproc}
%\usepackage[margin=1in]{geometry}
\usepackage{geometry}
\usepackage{amsmath, amsthm, amssymb}
\usepackage[colorlinks=true, pdfstartview=FitV, linkcolor=blue,citecolor=blue, urlcolor=blue]{hyperref}
\usepackage[abbrev,lite,nobysame]{amsrefs}
\usepackage{times}
\usepackage[usenames,dvipsnames]{color}
%\usepackage{mathtools}
\usepackage{polski}
\usepackage[utf8]{inputenc}
%%%Citation keys in blue, small on the side
%\providecommand*\showkeyslabelformat[1]{{\normalfont \tiny#1}}
%\usepackage[notref,notcite,color]{showkeys}
%\definecolor{labelkey}{rgb}{0,0,1}

%\mathtoolsset{showonlyrefs=true}
%%Citation keys in blue, small on the side
\providecommand*\showkeyslabelformat[1]{{\normalfont \tiny#1}}
\usepackage[notref,notcite,color]{showkeys}
\definecolor{labelkey}{rgb}{0,0,1}
\usepackage{dsfont}
%%%%%%%%%%%%%%%%%%%%%%%%%%%%%%%%%%%%%%%%%%%%
% DEFS

\newcommand{\red}[1]{\textcolor{red}{#1}}
\newcommand{\green}[1]{\textcolor{green}{#1}}

\def\eps{\varepsilon}
\def\e{{\rm e}}
\def\dist{{\rm dist}}
\def\div{{\rm div}}
\def\dd{{\rm d}}
\def\ddt{{\frac{\dd}{\dd t}}}
\def\IN {{\mathds 1}}
\def\Id{{\rm Id}}
\def\th{\vartheta}
\def\I {\mathbb{I}}
\def\R {\mathbb{R}}
\def\u {\boldsymbol{u}}
\def\v {\boldsymbol{v}}
\def\w {\boldsymbol{w}}
\def\x {\boldsymbol{x}}
\def\EE {\mathbb{E}}
\def\N {\mathbb{N}}
\def\H {{\mathcal H}}
\def\V {{\mathcal V}}
\def\B {{\mathcal B}}
\def\D {{\mathcal D}}
\def\E {{\mathcal E}}
\def\AA {{\mathcal A}}
\def\K {{\mathcal K}}
\def\KK {{\mathbb K}}
\def\J {{\mathcal J}}
\def\JJ {{\mathbb J}}
\def\SS {{\mathbb S}}
\def\II {{\mathcal I}}
\def\RR {{\mathcal R}}
\def\F {{\mathcal F}}
\def\X {{\mathcal X}}
\def\L {{\mathbb L}}
\def\Q {{\mathcal Q}}
\def\O {{\mathcal O}}
\def\q {{\tilde q}}

\def\NN {{\mathcal N}}
\def\M {{\mathcal M}}
\def\W {{\mathcal W}}
\def\P {{\mathbb P}}
\def\ZZ {{\mathbb Z}}
\def\S {{\mathbb S}}
\def \r {\rangle}
\def\T {{\mathbb T}}
\def\TT {{\mathbb T}^2}
\def\C {{\mathcal C}}
\DeclareMathOperator*{\esup}{ess\,sup}
\def\de{{\partial}}

%%%%%%%%%%%%%%%%%%%%%%%%%%%%%%%%%%%%%%%%%%%%
\newtheorem{proposition}{Proposition}[section]
\newtheorem{theorem}[proposition]{Theorem}
\newtheorem{corollary}[proposition]{Corollary}
\newtheorem{lemma}[proposition]{Lemma}
\theoremstyle{definition}
\newtheorem{definition}[proposition]{Definition}
\newtheorem{remark}[proposition]{Remark}
\newtheorem{example}[proposition]{Example}
\numberwithin{equation}{section}
%%%%%%%%%%%%%%%%%%%%%%%%%%%%%%%%%%%%%%%%%%%%
\begin{document}

\begin{center}
\begin{large}
	Dynamika dla słabo tłumionego równania falowego
	\end{large}
\end{center}

\medskip
\medskip

W ramach pracy doktorskiej będzie badana zagadnienie brzegowo początkowe dla równania
$$
\varepsilon u_{tt} + u_t - \Delta u = f(t,u),
$$
gdzie poszukiwana funkcja $u:\Omega \to \mathbb{R}$ jest określona na zbiorze otwartym i ograniczonym $\Omega\subset \mathbb{R}^3$ o gładkim brzegu.
Dla równania dodatkowo zakładamy warunek brzegowy $u(x,t) = 0$ dla $x\in \partial \Omega$ i warunki początkowe $u(t_0)=u_0$ oraz $u_t(t_0) = u_1$.  

Klasyczne wyniki dotyczące atraktorów globalnych i jednostajnych dla powyższego zagadnienia znajdują się w monografiach Babina i Visika \cite{Babin-Vishik-1992} oraz Chepyzhova i Visika \cite{Chepyzhov-Vishik-2002}. W szczególności istnienie atraktora globalnego (dla przypadku gdy $f$ nie zależy od $t$) i atraktorów pullback/jednostajnego/kocyklu (dla przypadku gdy $f$ zależy od $t$) było od dawna znane dla przypadku gdy $f$ spełnia sześcienny warunek wzrostu $|f(t,s)|\leq C(1+|s|^3)$ \cite{Arrieta-Carvalho-Hale-1992}. W ostatnich latach Savostianov, Kalantarov i Zelik \cite{Savostianov}, z wykorzystaniem tzw. oszacowań Strichartza w przestrzeni $L^4(0,T;L^{12}(\Omega))$ wykazali istnienie atraktora globalnego dla przypadku $f$ niezależnego od $t$ spełniającego warunek wzrostu piątego stopnia $|f(s)|\leq C(1+|s|^5)$, tym samym przesuwając krytyczny wykładnik o $2$. 

W ramach pracy planowane są dalsze badania nad zagadnieniem z warunkiem wzrostu piątego stopnia. W pierwszym kroku zajmiemy się przypadkiem $\epsilon=1$ i przedstawimy ekwiwalent wyniku Savostianova, Kalantarova i Zelika z pracy \cite{Savostianov} dla przypadku, gdy $f$ zależy od $t$. Istnieje kilka teorii atraktorów dla przypadku nieautonomicznego, które zostały niedawno zunifikowane ze sobą w pracy Bortolana, Carvalho i Langi \cite{Bortolan-Carvalho-Langa-2014}. W pierwszym kroku ramach pracy doktorskiej planowane jest uzyskanie wyników o istnieniu atraktorów nieautonomicznych: pullback/jednostajnego/kocyklu dla warunku wzrostu stopnia $5$ i podanie twierdzeń o relacjach między tymi atraktorami w duchu pracy \cite{Bortolan-Carvalho-Langa-2014}. 

W kolejnym kroku planowane jest badanie zbieżności atraktorów globalnych dla zagadnień z $\epsilon>0$ do atraktora dla zagadnienia z $\epsilon=0$. Problem jest trudny, gdyż równanie wówczas zmienia swój charakter z hiperbolicznego na paraboliczny. Ponadto nie jest możliwe  proste uogólnienie znanych wyników Hale'a i Raugel \cite{Hale-Raugel-1988, Hale-Raugel-1990} dla przypadku autonomicznego dotyczących tego typu zbieżności, gdyż zadanie to wymaga szczegółowej analizy tego jak stała w oszacowaniach Strichartza zależy od liczby $\epsilon$. Planowany jest dowód dolnie i górnie półciągłej w sensie metryki Hausdorffa zbieżności atraktorów globalnych (co uogólnia wyniki Hale'a i Raugel \cite{Hale-Raugel-1988, Hale-Raugel-1990}) oraz atraktorów nieautonomicznych (co uogólnia wyniki Freitasa, Kality i Langi \cite{Fre_Kal_La}) na przypadek warunku wzrostu z wykładnikiem $5$. 
Wydaje się że wynik może dać się uzyskać z wykorzystaniem oszacowań Smitha i Sogge \cite{Smith_Sogge} na tak zwane klastry spektralne. 

Trzecim, i ostatnim, zagadnieniem badanym w kontekście słabo tłumionego równanie falowego będą rozmaitości inercyjne i warunki stożka dla tego równania (wersja autonomiczna z $\varepsilon=1$). Przy odpowiednich założeniach na $f$ (niezależnego od $t$) wiadomo, że równanie to posiada rozmaitość inercyjną, będącą wykresem funkcji Lipschitzowskiej nad skończenie wymiarową przestrzenią, zawierającym w sobie globalny atraktor, przyciągającym wykładniczo wszystkie trajektorie, i dodatnio półniezmienniczym. W pracy Zgliczyńskiego i Capińskiego \cite{cap}
zaproponowano tak zwane warunki stożka wyższego rzędu. Warunki te gwarantują gładkość $C^k$ różnego rodzaju rozmaitości niezmienniczych dla równań zwyczajnych. Planowane jest zbadanie czy i w jaki sposób warunki te można przenieść na przypadek nieskończenie wymiarowy (wiadomo bowiem, że metoda transformacji wykresu na której bazuje konstrukcja rozmaitości niezmienniczych przenosi się na przypadek nieskończenie wymiarowy). Jeśli będzie to możliwe, doprowadzi to do dowodu nowego wyniku, że rozmaitość inercyjna dla badanego równania (o której wiadomo, że jest Lipschitzowska) jest również klasy $C^k$. 

\bibliography{wave_bib}

\end{document}
